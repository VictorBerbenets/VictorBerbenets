\documentclass[letterpaper,11pt]{article}

\usepackage{latexsym}
\usepackage{mismath}
\usepackage{physics}
\usepackage[empty]{fullpage}
\usepackage{titlesec}
\usepackage{marvosym}
\usepackage[usenames,dvipsnames]{color}
\usepackage{verbatim}
\usepackage{enumitem}
\usepackage[hidelinks]{hyperref}
\usepackage{fancyhdr}
\usepackage[english]{babel}
\usepackage{tabularx}
\usepackage{xcolor}
\usepackage{fontawesome5}

\input{glyphtounicode}

% -------------------- FONT OPTIONS --------------------
% sans-serif
% \usepackage[sfdefault]{roboto}
% \usepackage[sfdefault]{noto-sans}
% serif
% \usepackage{charter}

\pagestyle{fancy}
\fancyhf{} % clear all header and footer fields
\fancyfoot{}
\renewcommand{\headrulewidth}{0pt}
\renewcommand{\footrulewidth}{0pt}

% Adjust margins
\addtolength{\oddsidemargin}{-0.5in}
\addtolength{\evensidemargin}{-0.5in}
\addtolength{\textwidth}{1in}
\addtolength{\topmargin}{-1in} % Default was -.5in
\addtolength{\textheight}{1.0in}

\urlstyle{same}

\raggedbottom
\raggedright
\setlength{\tabcolsep}{0in}

% Section formatting
\titleformat{\section}{
  \vspace{-5pt}\scshape\raggedright\large
}{}{0em}{}[\color{black}\titlerule \vspace{-5pt}]

% Subsection formatting
\titleformat{\subsection}{
  \vspace{-4pt}\scshape\raggedright\large
}{\hspace{-.15in}}{0em}{}[\color{black}\vspace{-8pt}]

% Ensure that generate pdf is machine readable/ATS parsable
\pdfgentounicode=1

% -------------------- CUSTOM COMMANDS --------------------
\newcommand{\resumeItem}[1]{
  \item\small{
    {#1 \vspace{-2pt}}
  }
}

\newcommand{\resumeSubheading}[4]{
  \vspace{-2pt}\item
    \begin{tabular*}{0.97\textwidth}[t]{l@{\extracolsep{\fill}}r}
      \textbf{#1} & #2 \\
      \textit{\small#3} & \textit{\small #4} \\
    \end{tabular*}\vspace{-7pt}
}

\newcommand{\resumeSubSubheading}[2]{
    \item
    \begin{tabular*}{0.97\textwidth}{l@{\extracolsep{\fill}}r}
      \textit{\small#1} & \textit{\small #2} \\
    \end{tabular*}\vspace{-7pt}
}

\newcommand{\resumeProjectHeading}[2]{
    \item
    \begin{tabular*}{0.97\textwidth}{l@{\extracolsep{\fill}}r}
      \small#1 & #2 \\
    \end{tabular*}\vspace{-7pt}
}

\newcommand{\resumeSubItem}[1]{\resumeItem{#1}\vspace{-4pt}}
\newcommand{\resumeSubHeadingListStart}{\begin{itemize}[leftmargin=0.15in, label={}]}
\newcommand{\resumeSubHeadingListEnd}{\end{itemize}}
\newcommand{\resumeItemListStart}{\begin{itemize}}
\newcommand{\resumeItemListEnd}{\end{itemize}\vspace{-5pt}}

\renewcommand\labelitemii{$\vcenter{\hbox{\tiny$\bullet$}}$}

\setlength{\footskip}{4.08003pt}

% -------------------- START OF DOCUMENT --------------------
\begin{document}

% -------------------- HEADING--------------------
\
\vspace{+5pt}

\begin{center}
    \textbf{\Huge \scshape Victor Berbenets} \\ \vspace{8pt}
    \small 
    \faIcon{github}
    \href{https://github.com/VictorBerbenets}{\underline{github.com/VictorBerbenets}} $  $

    \faIcon{envelope}
    {\underline{berbenets.vd@phystech.edu}}
\end{center}

% -------------------- EDUCATION --------------------
\section{Education}
  \resumeSubHeadingListStart
  
    \resumeSubheading
      {MIPT(Moscow Institute of Physics and Thechnology)}{2022 - 2026}
      {DREC(Department of Radio Engineering and Cybernatics)}{Current GPA: 7.09/10}
      

    \vspace{+10pt}

    \subsection{Coursework}
      \textbf{Main Courses:} Discrete Math, Linear Algebra, Calculus, Physics,
      Computer technologies, Uses and Applications of C++ (by K. I. Vladimirov),
      Compiler Technologies and Professional Programming (by I. R. Dedinsky)\\


  \resumeSubHeadingListEnd
\section{Languages}
 \begin{itemize}[leftmargin=0.15in, label={}]
    \begin{enumerate}
        \item Russian / Native.
        \item English / B1-B2.
    \end{enumerate}
 \end{itemize}
% -------------------- SKILLS --------------------
\section{Skills}
 \begin{itemize}[leftmargin=0.15in, label={}]
    \small{\item{
    
     \textbf{Programming Languages}{: C/C++, Python, x86 Assembler} \\
     
     \textbf{Tools}{: Cmake, Git/GitHub, Linux, Vim, gdb, valgrind, VS Code, Latex}
     
     % \textbf{Frameworks}{: React, Node.js, Flask, JUnit, WordPress, Material-UI, FastAPI} \\
     
     % \textbf{Libraries}{: pandas, NumPy, Matplotlib}
     
    }}
 \end{itemize}

% -------------------- PROJECTS --------------------
\section{Best Projects}
    \resumeSubHeadingListStart

        \resumeProjectHeading
        {\textbf{Cache} $|$ \footnotesize\emph{C++, Cmake, Git, Bash, Vim} }{Aug. 2023}
        \resumeItemListStart
            \resumeItem{LFU (least frequently used) cache realization.}
            \resumeItem{My first introduction to basic language concepts.}
          \resumeItemListEnd
    
        \resumeProjectHeading
        {\textbf{Triangles} $|$ \footnotesize\emph{C++, Python, Cmake, Gtest, Git, Bash, Vim}}{Sep. 2023}
        \resumeItemListStart
            \resumeItem{Detecting intersecting triangles in 3D space.}
            \resumeItem{Building your own hierarchy of geometric figures: from a point and a segment to a triangle and a plane.}
          \resumeItemListEnd
      \resumeProjectHeading
        {\textbf{AVL Tree} $|$ \footnotesize\emph{C++, Graphviz, Cmake, Gtest, Git, Bash, Vim}}{Oct. 2023}
        \resumeItemListStart
            \resumeItem{Implementation of an AVL-based search tree with lower-upper bound methods and Iterators.}
            \resumeItem{The main task, besides creating a tree, is to write an effective distance method that works in O(logn).}
        \resumeItemListEnd
    \resumeProjectHeading
        {\textbf{Matrix} $|$ \footnotesize\emph{C++, Cmake, Gtest, Git, Bash, Vim}}{Oct. 2023}
        \resumeItemListStart
            \resumeItem{Implementation of a matrix class with a method for calculating the determinant and Contiguous Iterator class.}
        \resumeItemListEnd
    \resumeProjectHeading
        {\textbf{MatrixChain} $|$ \footnotesize\emph{C++, Cmake, Gtest, Git, Bash, Vim}}{Nov. 2023}
        \resumeItemListStart
            \resumeItem{Efficient matrix chain multiplication.}
            \resumeItem{The program prints the optimal sequence of multiplying matrix pairs.}
        \resumeItemListEnd
    \resumeProjectHeading
        {\textbf{ParaCL} $|$ \footnotesize\emph{C++, Bison, Flex, Cmake, Gtest, Git, Bash, Vim}}{Dec. 2023 - Feb. 2024}
        \resumeItemListStart
            \resumeItem{Custom C language (Interpreter). Frontend was implemented with Flex \& Bison.}
            \resumeItem{A large project that helped to thoroughly study the basic principles of OOP.}
        \resumeItemListEnd
    \resumeProjectHeading
        {\textbf{Graph} $|$ \footnotesize\emph{C++, Cmake, Git, Bash, Vim}}{Feb. 2024 - Mar. 2024}
        \resumeItemListStart
            \resumeItem{Donald Knut style representation of a graph (TAOCP 7.1). The entire graph is built \\ on one std::vector, which stores std::variants.}
            \resumeItem{The program checks the graph for bipartiteness: if the graph is bipartite,\\ it displays its colored representation, otherwise it displays a cycle of odd length}
        \resumeItemListEnd
    \resumeProjectHeading
        {\textbf{OfflineLCA} $|$ \footnotesize\emph{C++, Cmake, Git, Bash, Vim}}{Mar. 2024}
        \resumeItemListStart
            \resumeItem{Solving RMQ problem in O(n) preprocessing and answering each query in O(1)}
        \resumeItemListEnd
    \resumeProjectHeading
        {\textbf{Differentiator} $|$ \footnotesize\emph{C, Graphviz, Latex, make, Git, VS Code}}{Mar. 2023 - Apr. 2023}
        \resumeItemListStart
            \resumeItem{A program that calculates derivatives with respect to all variables and outputs it all to a pdf file}
        \resumeItemListEnd
    \resumeProjectHeading
        {\textbf{CPU} $|$ \footnotesize\emph{C, make, Git, VS Code}}{Nov. 2022 - Jan. 2023}
        \resumeItemListStart
            \resumeItem{Assembler: support for basic proprietary commands.}
            \resumeItem{Virtual processor: executes programs written on proprietary assembler.}
        \resumeItemListEnd
    \resumeSubHeadingListEnd
    

\end{document}

